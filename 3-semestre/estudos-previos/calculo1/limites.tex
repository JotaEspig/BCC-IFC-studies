\documentclass[12pt,a4paper,brazil]{article}

\usepackage[T1]{fontenc}
\usepackage{hyperref}
\usepackage{graphicx}
\usepackage{geometry}
\usepackage{amsmath}
\usepackage{amssymb}
\usepackage{blindtext}
\usepackage{outlines}
\usepackage[brazil]{babel}

\hypersetup {
	colorlinks=true,
	urlcolor=blue,
}

\newcommand{\minus}{\scalebox{0.75}[1.0]{$-$}}

\title{Limites}
\author{Jota}
\date{\today}

\begin{document}

\maketitle

\section{\textbf{Limites}}
Considere a função $f(x) = \frac{x^2 \minus 4}{x\minus2}$.
Não temos um valor de $f(x)$ quando $x=2$, pois não é possível dividir por 0.
O limite dessa função com $x$ se aproximando de 2 é o valor que $f(x)$ tende conforme
a aproximação de $x$ chega mais próxima do 2, i.e. 4. É possível visualizar isto através da tabela:
\begin{center}
	\begin{tabular}{l|l}
		$x$       	   & $f(x)$  \\ \hline
		$\minus2.1$    & $4.1$   \\
		$\minus2.01$   & $4.01$  \\
		$\minus2.001$  & $4.001$ \\
		$\minus2.0001$ & $4.0001$ 
	\end{tabular} \\
	Conforme o $x$ se aproxima de 2, $f(x)$ se aproxima de 4
\end{center}
Usando os símbolos de limite: $\lim\limits_{x \to 2} \frac{x^2 \minus 4}{x\minus2} = 4$

Podemos ter funções sem limites definidos, e.g. $\sin(1/x)$. É possível visualizar isto usando a tabela:
\begin{center}
	\begin{tabular}{l|l}
		$x$       & $\sin(1/x)$     \\ \hline
		$\minus0.1$    & $0.54402111$  \\
		$\minus0.01$   & $0.50636564$  \\
		$\minus0.001$  & $\minus0.99388865$ \\
		$\minus0.0001$ & $\minus0.99388865$ 
	\end{tabular} \\
	Veja o \href{https://www.mathway.com/pt/popular-problems/Calculus/501510}{link}
\end{center}

\subsection{\textbf{Limites laterais}}
Considere a função \[f(x) =
	\begin{cases}
		x < 2 \Rightarrow x + 1 \\
		x \ge 2 \Rightarrow x^2 \minus 4
	\end{cases}
\]

O limite $\lim\limits_{x \to 2} f(x)$ não existe, pois se aproximarmos de 2 pelo ``lado esquerdo'' encontrariamos \minus1
e pelo ``lado direito'' encontrariamos 2, para essa função temos os chamados limites laterais:
\[\lim\limits_{x \to 2^{\minus}} f(x)\]
\begin{center}
	representa a aproximação pela ``esquerda'' (negativo)
\end{center}
\[\lim\limits_{x \to 2^{+}} f(x)\]
\begin{center}
	representa a aproximação pela ``direita'' (positivo)
\end{center}

\subsection{\textbf{Limites infinitos}}
Considere a função $f(x) = \frac{1}{x}$, ela tem limites infinitos:
\[\lim\limits_{x \to 0^{\minus}} f(x) = \minus\infty\]
\[\lim\limits_{x \to 0^{+}} f(x) = +\infty\]

\subsection{\textbf{Leis do limite}}
\underline{Teorema: Limites básicos} \\
Qualquer número real $a$ ou constante $c$
\begin{center}
	\begin{enumerate}
		\item $\lim\limits_{x \to a} x = a$
		\item $\lim\limits_{x \to a} c = c$
	\end{enumerate}
\end{center}
\underline{Teorema: Leis do limite} \\
Seja $f(x)$ e $g(x)$ definidas para todo $x \neq a$ em um intervalo aberto contendo $a$.
Seja $c$ uma constante. Então temos que:
\begin{center}
	\begin{enumerate}
		\item \textbf{Lei da soma de limites}:
			$\lim\limits_{x \to a} [f(x) + g(x)] = \lim\limits_{x \to a} f(x) + \lim\limits_{x \to a} g(x)$
		\item \textbf{Lei da diferença de limites}:
			$\lim\limits_{x \to a} [f(x) \minus g(x)] = \lim\limits_{x \to a} f(x) \minus \lim\limits_{x \to a} g(x)$
		\item \textbf{Lei da multiplicação constante de limites}:
			$\lim\limits_{x \to a} [c f(x)] = c \lim\limits_{x \to a} f(x)$
		\item \textbf{Lei do produto de limites}:
			$\lim\limits_{x \to a} [f(x) \times g(x)] = \lim\limits_{x \to a} f(x) \times \lim\limits_{x \to a} g(x)$
		\item \textbf{Lei do quociente de limites}:
			$\lim\limits_{x \to a} \frac{f(x)}{g(x)} = \frac{\lim\limits_{x \to a} f(x)}{\lim\limits_{x \to a} g(x)}$
		\item \textbf{Lei da potência de limites}:
			$\lim\limits_{x \to a} (f(x))^n = (\lim\limits_{x \to a} f(x))^n$ para todo $n$ positivo
		\item \textbf{Lei da raiz de limites}:
			$\lim\limits_{x \to a} \sqrt[n]{f(x)} = \sqrt[n]{\lim\limits_{x \to a} f(x)}$ para todo $n$ positivo
	\end{enumerate}
\end{center}
\textbf{Resolvendo um exemplo utilizando as leis acima:} \\
Tente resolver o limite proposto: $\lim\limits_{x \to \minus3} (4x + 2)$ \\
\textbf{Solução:}
\begin{equation*}
	\begin{aligned}
		\lim\limits_{x \to \minus3} (4x + 2)
		&= \lim\limits_{x \to \minus3} 4x + \lim\limits_{x \to \minus3} 2 & & \text{(Lei da soma de limites)} \\
		&= 4 \times \lim\limits_{x \to \minus3} x + 2 & & \text{(Lei da multiplicação constante de limites)} \\
		&= 4 \times (\minus3) + 2 \\
		&= \minus10
	\end{aligned}
\end{equation*}
\textbf{Avaliando funções polinomiais e racionais} \\
É visível que dado uma função $f(x)$, seu limite se dará como:
$\lim\limits_{x \to a} f(x) = f(a)$. Porém nem todos os casos isso é válido. \\
Perceba que dado duas funções $p(x)$ e $q(x)$, o limite se da como:
$\lim\limits_{x \to a} \frac{p(x)}{q(x)} = \frac{p(a)}{q(a)}$, quando $q(a) \neq 0$.
\vspace{5mm} \\ % 5mm vertical space
\textbf{Técnicas adicionais para avaliar um limite} \\
Considere a função $f(x) = \frac{x^2 - 1}{x - 1}$, o limite $\lim\limits_{x \to 1} f(x)$
não pode ser calculado utilizando as leis do limite citadas. Esse limite tem a forma de
$\lim\limits_{x \to a} \frac{f(x)}{g(x)}$, onde $\lim\limits_{x \to a} f(x) = 0$ e $\lim\limits_{x \to a} g(x) = 0$.
Nesse caso falamos que $\frac{f(x)}{g(x)}$ tem a forma indeterminada de $\frac{0}{0}$.
O passo a passo a seguir contém estratégias de como conseguir solucionar esse tipo de limite.
\begin{outline}[enumerate]
	\1 Identificar que não é possível utilizar apenas as leis do limite
	\1 Encontrar uma função igual a $h(x) = \frac{f(x)}{g(x)}\,\forall x \neq a$ em algum intervalo que contém $a$
	\2 Se $f(x)$ e $g(x)$ forem polinomiais, devemos fatorar cada função e eliminar qualquer fator comum
	\2 Se o numerador ou o denominador contêm uma diferença envolvendo raiz quadrada, devemos tentar multiplicar
	o numerador e o denominador pelo conjugado da expressão contendo a raiz
	\2 Se $\frac{f(x)}{g(x)}$ é uma função complexa, devemos tentar simplificar-la
	\1 E depois, aplicamos as leis do limite
\end{outline}
\vspace{5mm} % 5mm vertical space
\textbf{Resolva}: $\lim\limits_{x \to 3} (\frac{1}{x - 3} - \frac{4}{x^2 - 2x - 3})$
\vspace{3mm} \\ % 5mm vertical space
\textbf{Solução}:
\begin{equation*}
	\begin{aligned}
		\lim\limits_{x \to 3} (\frac{1}{x - 3} - \frac{4}{x^2 - 2x - 3})
		&= \lim\limits_{x \to 3} (\frac{x + 1}{x + 1}\cdot\frac{1}{x - 3} - \frac{4}{x^2 - 2x - 3}) \\
		&= \lim\limits_{x \to 3} (\frac{x + 1}{x^2 - 2x - 3} - \frac{4}{x^2 - 2x - 3}) \\
		&= \lim\limits_{x \to 3} (\frac{x + 1 - 4}{x^2 - 2x - 3}) \\
		&= \lim\limits_{x \to 3} (\frac{x - 3}{x^2 - 2x - 3}) \\
		&= \lim\limits_{x \to 3} (\frac{x - 3}{(x + 1)(x - 3)}) \\
		&= \lim\limits_{x \to 3} (\frac{1}{x + 1}) \\
		&= \frac{1}{3 + 1} \\
		&= \frac{1}{4}
	\end{aligned}
\end{equation*}

\end{document}
