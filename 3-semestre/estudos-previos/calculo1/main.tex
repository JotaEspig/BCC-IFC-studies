\documentclass[12pt,a4paper]{article}
\usepackage[T1]{fontenc}
\usepackage[brazilian]{babel}
\usepackage{hyperref}
\usepackage{graphicx}
\usepackage{amsmath}

\hypersetup {
	colorlinks=true,
	urlcolor=blue,
}

\newcommand{\minus}{\scalebox{0.75}[1.0]{$-$}}

\begin{document}

\section*{Limites}
Considere a função $f(x) = \frac{x^2 - 4}{x-2}$.
Não temos um valor de $f(x)$ quando $x=2$, pois não é possível dividir por 0.
O limite dessa função com $x$ se aproximando de 2 é o valor que $f(x)$ tende conforme
a aproximação de $x$ chega mais próxima do 2, i.e. 4.
Usando os símbolos de limite: $\lim\limits_{x \to 2} \frac{x^2 - 4}{x-2} = 4$

Podemos ter funções sem limites definidos, e.g. $\sin(1/x)$. É possível visualizar isto usando a tabela:
\begin{center}
	\begin{tabular}{l|l}
		$x$       & $\sin(x)$     \\ \hline
		$\minus0.1$    & $0.54402111$  \\
		$\minus0.01$   & $0.50636564$  \\
		$\minus0.001$  & $\minus0.99388865$ \\
		$\minus0.0001$ & $\minus0.99388865$ 
	\end{tabular}
\end{center}
Veja o \href{https://www.mathway.com/pt/popular-problems/Calculus/501510}{link}

\subsection*{Limites laterais}
Considere a função \[f(x) =
	\begin{cases}
		x < 2 \Rightarrow x + 1 \\
		x \ge 2 \Rightarrow x^2 - 4
	\end{cases}
\]

O limite $\lim\limits_{x \to 2} f(x)$ não existe, pois se aproximarmos de 2 pelo ``lado esquerdo'' encontrariamos -1
e pelo ``lado direito'' encontrariamos 2, para essa função temos os chamados limites laterais:
\[\lim\limits_{x \to 2^{\minus}} f(x)\]
\begin{center}
	representa a aproximação pela ``esquerda'' (negativo)
\end{center}
\[\lim\limits_{x \to 2^{+}} f(x)\]
\begin{center}
	representa a aproximação pela ``direita'' (positivo)
\end{center}

\subsection*{Limites infinitos}
Considere a função $f(x) = \frac{1}{x}$, ela tem limites infinitos:
\[\lim\limits_{x \to 0^{\minus}} f(x) = \minus\infty\]
\[\lim\limits_{x \to 0^{+}} f(x) = +\infty\]

\section*{Derivadas}

\end{document}
