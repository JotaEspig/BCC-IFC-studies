\documentclass[12pt,a4paper,brazil]{article}

\usepackage{hyperref}
\usepackage{listings}
\usepackage{color}
\usepackage[T1]{fontenc}
\usepackage[brazil]{babel}
\usepackage{amsmath}

\definecolor{dkgreen}{rgb}{0,0.6,0}
\definecolor{gray}{rgb}{0.5,0.5,0.5}
\definecolor{mauve}{rgb}{0.58,0,0.82}

\lstset{
    frame=tb,
    language=Haskell,
    aboveskip=3mm,
    belowskip=3mm,
    showstringspaces=false,
    columns=flexible,
    basicstyle={\small\ttfamily},
    numbers=none,
    numberstyle=\tiny\color{gray},
    keywordstyle=\color{blue},
    commentstyle=\color{dkgreen},
    stringstyle=\color{mauve},
    breaklines=true,
    breakatwhitespace=true,
    tabsize=4
}

\title{Lista 05}
\author{João Vitor Espig}
\date{\today}

\begin{document}

\maketitle
Usando \LaTeX

\section*{Exercício 1}
O resultado é (False,-4).
O primeiro valor é False pois a expressão
$$1 == 4 \ \&\& \ \text{True}$$
é interpretada como
$$(1 == 4) \ \&\& \ \text{True}$$
que é falsa ($1 == 4$ resulta em False,
e consequentemente $\text{False} \ \&\& \ \text{True}$ resulta em False). \\
O segundo valor é -4 pois a expressão
$$\text{mod} \ (4*8) \ 31^2-5$$
é interpretada como
$$(\text{mod} \ (4*8) \ (31))^2-5$$
Logo:
\begin{align*}
(\text{mod} \ (4*8) \ (31))^2-5 & = (\text{mod} \ (32) \ (31))^2-5 \\
                                &= (1)^2 - 5 \\
                                &= -4
\end{align*}
\section*{Exercício 2}
4 e 0.5: \\
- 4\char94 0.5 resulta em um erro \\
- 4 ** (0.5) resulta em 2

\section*{Exercício 3}
\begin{lstlisting}
dobro :: Double -> Double
dobro x = 2 * x
\end{lstlisting}

\section*{Exercício 4}
\begin{lstlisting}
\end{lstlisting}

\section*{Exercício 5}
o comando nos mostra o tipo do valor que é passado após o :t, avaliando a expressão
"decremento (incremento 9)" \ temos 9, pois incremento 9 = 10 e decremento 10 = 9.
Assim o tipo de 9 é Num a => a.

\section*{Exercício 6}
\begin{lstlisting}
\end{lstlisting}

\section*{Exercício 7}
\begin{lstlisting}
\end{lstlisting}

\section*{Exercício 8}
\begin{lstlisting}
\end{lstlisting}

\section*{Exercício 9}
\begin{lstlisting}
negate (-9)
\end{lstlisting}

\section*{Exercício 10}
negate -9 resulta em um erro pois a expressão não é interpretada como negate (-9) e sim como (negate) (-) 9.

\end{document}
