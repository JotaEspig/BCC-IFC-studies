\documentclass[12pt,a4paper,brazil]{article}

\usepackage{hyperref}
\usepackage{listings}
\usepackage{color}
\usepackage[T1]{fontenc}
\usepackage[brazil]{babel}
\usepackage{amsmath}

\definecolor{dkgreen}{rgb}{0,0.6,0}
\definecolor{gray}{rgb}{0.5,0.5,0.5}
\definecolor{mauve}{rgb}{0.58,0,0.82}

\lstset{
    frame=tb,
    language=Haskell,
    aboveskip=3mm,
    belowskip=3mm,
    showstringspaces=false,
    columns=flexible,
    basicstyle={\small\ttfamily},
    numbers=none,
    numberstyle=\tiny\color{gray},
    keywordstyle=\color{blue},
    commentstyle=\color{dkgreen},
    stringstyle=\color{mauve},
    breaklines=true,
    breakatwhitespace=true,
    tabsize=4
}

\title{Lista 06}
\author{João Vitor Espig}
\date{\today}

\begin{document}

\maketitle
Usando \LaTeX

\section*{Exercício 1}
\begin{lstlisting}
import Text.Printf

areaCircunsferencia r = pi * r^2

main :: IO()
main = do
    let result = (let r = 10 :: Double in areaCircunsferencia r)
    printf "Area da circunferencia: %.2f\n" result
\end{lstlisting}

\section*{Exercício 2}
\begin{lstlisting}
tipoTriangulo a b c
    | a <= 0 || b <= 0 || c <= 0 = "NAOTRI"
    | a == b && b == c = "Equilatero"
    | a + b <= c || a + c <= b || b + c <= a = "NAOTRI"
    | a == b || a == c || b == c = "Isosceles"
    | otherwise = "Escaleno"
\end{lstlisting}

\section*{Exercício 3}
\begin{lstlisting}
multiplica x y
    | x == 0 || y == 0 = 0
    | y > 0 = x + multiplica x (y - 1)
    | y < 0 = -multiplica x (-y)
\end{lstlisting}

\section*{Exercício 4}
\begin{lstlisting}
multiplica x y
    | x == 0 || y == 0 = 0
    | y > 0 = x + multiplica x (y - 1)
    | y < 0 = -multiplica x (-y)

multiplicaNaturais x y
    | x < 0 || y < 0 = error "multiplicaNaturais: argumentos negativos"
    | otherwise = multiplica x y
\end{lstlisting}

\section*{Exercício 5}
\begin{lstlisting}
maxAux x y
    | x > y = x
    | otherwise = y

myMax x y z
    | x >= maxAux y z = x
    | y >= maxAux x z = y
    | z >= maxAux x y = z

myMin x y z = negate (myMax (-x) (-y) (-z))
\end{lstlisting}

\section*{Exercício 6}
\begin{lstlisting}
xor x y = x /= y
\end{lstlisting}

\section*{Exercício 7}
\begin{lstlisting}
clonaNumeros [] = []
clonaNumeros (x:xs) = x:x:clonaNumeros (xs)
\end{lstlisting}

\section*{Exercício 8}
\begin{lstlisting}
somaDoisPrimeiros (x:y:_) = x + y
somaDoisPrimeiros _ = error "somaDoisPrimeiros: lista com menos de dois elementos"
\end{lstlisting}

\section*{Exercício 9}
\begin{lstlisting}
listaAte x = [0..(abs x)]
\end{lstlisting}

\section*{Exercício 10}
\begin{lstlisting}
parOuImpar arr = map even arr
\end{lstlisting}

\section*{Exercício 11}
\begin{lstlisting}
soPar arr = filter even arr
\end{lstlisting}

\section*{Exercício 12}
\begin{lstlisting}
import Data.Char

soMinuscula arr = filter isLower arr
\end{lstlisting}

\section*{Exercício 13}
\begin{lstlisting}
import Data.Char

isVowel c = c `elem` "aeiouAEIOU"

substituiVogais :: [Char] -> [Char]
substituiVogais arr = map (\c -> if isVowel c then toUpper c else c) arr
\end{lstlisting}

\section*{Exercício 14}
\begin{lstlisting}
friboi arr = map (++ " Friboi") arr
\end{lstlisting}

\section*{Exercício 15}
\begin{lstlisting}
pertence x arr = x `elem` arr
\end{lstlisting}

\section*{Exercício 16}
\begin{lstlisting}
filtraAux [] newArr = newArr
filtraAux (x:xs) newArr = if x `elem` newArr
                            then filtraAux xs newArr
                            else filtraAux xs (x:newArr)

filtraLista arr = reverse (filtraAux arr [])
\end{lstlisting}

\section*{Exercício 17}
\begin{lstlisting}
nPrimeiros arr n = take n arr
\end{lstlisting}

\section*{Exercício 18}
\begin{lstlisting}
let arr = [x * 3 | x <- [0..100]]
\end{lstlisting}

\end{document}
