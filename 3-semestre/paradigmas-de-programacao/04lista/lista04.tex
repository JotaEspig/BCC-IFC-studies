\documentclass[12pt,a4paper,brazil]{article}

\usepackage{hyperref}
\usepackage{listings}
\usepackage{color}
\usepackage[T1]{fontenc}
\usepackage[brazil]{babel}

\definecolor{dkgreen}{rgb}{0,0.6,0}
\definecolor{gray}{rgb}{0.5,0.5,0.5}
\definecolor{mauve}{rgb}{0.58,0,0.82}

\lstset{
    frame=tb,
    language=Prolog,
    aboveskip=3mm,
    belowskip=3mm,
    showstringspaces=false,
    columns=flexible,
    basicstyle={\small\ttfamily},
    numbers=none,
    numberstyle=\tiny\color{gray},
    keywordstyle=\color{blue},
    commentstyle=\color{dkgreen},
    stringstyle=\color{mauve},
    breaklines=true,
    breakatwhitespace=true,
    tabsize=4
}

\title{Lista 04}
\author{João Vitor Espig}
\date{\today}

\begin{document}

\maketitle
Usando \LaTeX

\section*{Exercício 1}
\begin{lstlisting}
pessoa(joao).
pessoa(maria).
pessoa(pedro).
pessoa(marcos).
pessoa(joana).
pessoa(ricardo).
pessoa(bruno).

pai(joao, maria).
pai(joao, pedro).
pai(joao, marcos).
pai(joao, joana).
pai(pedro, ricardo).
pai(pedro, bruno).
\end{lstlisting}

\section*{Exercício 2}
\begin{lstlisting}
?- pai(X, ricardo).
\end{lstlisting}

\section*{Exercício 3}
\begin{lstlisting}
pessoa(joao).
pessoa(maria).
pessoa(pedro).
pessoa(marcos).
pessoa(joana).
pessoa(ricardo).
pessoa(bruno).
pessoa(victor).

% sexo(X, S) significa X eh do sexo S
sexo(joao, m).
sexo(maria, f).
sexo(pedro, m).
sexo(marcos, m).
sexo(joana, f).
sexo(ricardo, m).
sexo(bruno, m).
sexo(victor, m).

% pai(X, Y) significa X eh pai de Y
pai(joao, maria).
pai(joao, pedro).
pai(joao, marcos).
pai(joao, joana).
pai(pedro, ricardo).
pai(pedro, bruno).
pai(bruno, victor).

% irmao(X, Y) significa X eh irmao de Y
irmao(X, Y) :- pai(Z, X), pai(Z, Y), sexo(X, m), X \== Y.

% irma(X, Y) significa X eh irma de Y
irma(X, Y) :- pai(Z, X), pai(Z, Y), sexo(X, f), X \== Y.
\end{lstlisting}

\section*{Exercício 4}
\begin{lstlisting}
?- irma(X, bruno).
\end{lstlisting}

\section*{Exercício 5}
\begin{lstlisting}
pessoa(joao).
pessoa(maria).
pessoa(pedro).
pessoa(marcos).
pessoa(joana).
pessoa(ricardo).
pessoa(bruno).
pessoa(victor).

% sexo(X, S) significa X eh do sexo S
sexo(joao, m).
sexo(maria, f).
sexo(pedro, m).
sexo(marcos, m).
sexo(joana, f).
sexo(ricardo, m).
sexo(bruno, m).
sexo(victor, m).

% pai(X, Y) significa X eh pai de Y
pai(joao, maria).
pai(joao, pedro).
pai(joao, marcos).
pai(joao, joana).
pai(pedro, ricardo).
pai(pedro, bruno).
pai(bruno, victor).

% irmao(X, Y) significa X eh irmao de Y
irmao(X, Y) :- pai(Z, X), pai(Z, Y), sexo(X, m), X \== Y.

% irma(X, Y) significa X eh irma de Y
irma(X, Y) :- pai(Z, X), pai(Z, Y), sexo(X, f), X \== Y.

% neto(X, Y) significa X eh neto de Y
neto(X, Y) :- pai(Y, Z), pai(Z, X), sexo(X, m).

% neta(X, Y) significa X eh neta de Y
neta(X, Y) :- pai(Y, Z), pai(Z, X), sexo(X, f).

% comando para descobrir neto de joao: ?- neto(X, joao).
\end{lstlisting}

\section*{Exercício 6}
\begin{lstlisting}
pessoa(joao).
pessoa(maria).
pessoa(pedro).
pessoa(marcos).
pessoa(joana).
pessoa(ricardo).
pessoa(bruno).
pessoa(victor).

% sexo(X, S) significa X eh do sexo S
sexo(joao, m).
sexo(maria, f).
sexo(pedro, m).
sexo(marcos, m).
sexo(joana, f).
sexo(ricardo, m).
sexo(bruno, m).
sexo(victor, m).

% pai(X, Y) significa X eh pai de Y
pai(joao, maria).
pai(joao, pedro).
pai(joao, marcos).
pai(joao, joana).
pai(pedro, ricardo).
pai(pedro, bruno).
pai(bruno, victor).

% irmao(X, Y) significa X eh irmao de Y
irmao(X, Y) :- pai(Z, X), pai(Z, Y), sexo(X, m), X \== Y.

% irma(X, Y) significa X eh irma de Y
irma(X, Y) :- pai(Z, X), pai(Z, Y), sexo(X, f), X \== Y.

% neto(X, Y) significa X eh neto de Y
neto(X, Y) :- pai(Y, Z), pai(Z, X), sexo(X, m).

% neta(X, Y) significa X eh neta de Y
neta(X, Y) :- pai(Y, Z), pai(Z, X), sexo(X, f).

% bisneto(X, Y) significa X eh bisneto de Y
bisneto(X, Y) :- pai(Z, X), neto(Z, Y), sexo(X, m).

% bisneta(X, Y) significa X eh bisneta de Y
bisneta(X, Y) :- pai(Z, X), neto(Z, Y), sexo(X, f).

% comando para descobrir bisneto de joao: ?- bisneto(X, joao).
\end{lstlisting}

\section*{Exercício 7}
\begin{lstlisting}
aluno(joao).
aluno(maria).
aluno(ana).
aluno(antonio).
\end{lstlisting}

\section*{Exercício 8}
\begin{lstlisting}
aluno(joao).
aluno(maria).
aluno(ana).
aluno(antonio).

nota(joao, 10.0).
nota(maria, 9.0).
nota(ana, 8.0).
nota(antonio, 7.0).
\end{lstlisting}

\section*{Exercício 9}
\begin{lstlisting}
aluno(joao).
aluno(maria).
aluno(ana).
aluno(antonio).

nota(joao, 10.0).
nota(maria, 7.0).
nota(ana, 3.0).
nota(antonio, 6.9).

passou(X) :- aluno(X), nota(X, Z), Z >= 7.
\end{lstlisting}

\section*{Exercício 10}
\begin{lstlisting}
aluno(joao).
aluno(maria).
aluno(ana).
aluno(antonio).

nota(joao, 10.0).
nota(maria, 7.0).
nota(ana, 3.0).
nota(antonio, 6.9).

frequencia(joao, 90).
frequencia(maria, 75).
frequencia(ana, 50).
frequencia(antonio, 70).

passou(X) :- aluno(X), nota(X, Z), Z >= 7.
passou2(X) :- aluno(X), nota(X, Z), Z >= 7, frequencia(X, Y), Y >= 75.
\end{lstlisting}

\end{document}
